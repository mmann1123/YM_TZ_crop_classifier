\documentclass{article}
\usepackage[left=1in, right=1in, top=1in, bottom=1in]{geometry}
\usepackage{url}
\begin{document}

\noindent
Michael L. Mann \\
The George Washington University \\
Washington, DC 20052 \\
Email: mmann1123@gwu.edu \\
3/5/2025  

\vspace{1cm}
 
\noindent
\textbf{Subject:} Submission of Manuscript: \textit{``Lite Learning: Efficient Crop Classification in Tanzania Using Feature Extraction with Machine Learning \& Crowd Sourcing''}

\vspace{0.5cm}
\noindent
Dear Editor,  

I am pleased to submit our manuscript, \textit{``Lite Learning: Efficient Crop Classification in Tanzania Using Feature Extraction with Machine Learning \& Crowd Sourcing,''} for consideration in \textit{JSTARS}. This study introduces a \textbf{novel time-series feature extraction approach} for crop classification using \texttt{xr\_fresh}, a new Python module designed for geospatial time-series analysis.  

\vspace{0.3cm}

\noindent \textbf{Key Contributions:}
\begin{itemize}
    \item \texttt{xr\_fresh} automates the extraction of \textbf{50+ time-series metrics} (e.g., trend, variability, complexity) from Sentinel-2 imagery, optimizing temporal data for machine learning.
    \item Our approach \textbf{outperforms traditional land cover models}, achieving \textbf{Cohen’s Kappa of 0.82 and F1-micro score of 0.85}, while remaining computationally efficient.
    \item We integrate \textbf{YouthMappers crowdsourced field data} with time-series features to improve model generalization in \textbf{data-scarce agricultural regions}.
\end{itemize}

\noindent
This methodology offers a  scalable, interpretable alternative to deep learning in low-data environments, making it highly relevant to \textit{JSTARS}' focus on remote sensing applications for agriculture and food security.  We appreciate your time and consideration and look forward to your feedback.  

\vspace{0.5cm}
\noindent
Sincerely,  

\vspace{0.5cm}
\noindent
Michael L. Mann, Ph.D. \\
The George Washington University \\
Washington, DC 20052 \\
mmann1123@gwu.edu \\
\vspace{2.5cm}

\noindent
See my free textbook on geospatial python: \url{https://pygis.io}.

\end{document}
